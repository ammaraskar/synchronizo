\documentclass{article}
\usepackage[utf8]{inputenc}
\usepackage[margin=0.75in]{geometry}
\title{\textbf{Project Charter}\\ 
Synchronizo \\
\large Team \#29}
\author{Abhijeet Chakrabarti, Ammar Askar, Brian Quinn, Eric Lee }
\date{20 January 2017}

\begin{document}

\maketitle

\section{Problem Statement}
The music streaming industry has been surging to new heights, growing 76\% in the U.S. in 2016. Synchronizo aims to bring people together to enjoy listening to music in real-time. You can be anywhere in the world and still, listen to music with your friends who are miles away. This will be different from current music streaming services where you can share songs passively but not listen to them with people as if they were in the same room as you.
\section{Project Objectives}

\begin{itemize}
    \item Enable real-time listening of music across the Internet.
    \item Create a nice looking interface that is easy and intuitive to use.
    \item Add a social networking aspect allowing for real time chat, having ''friends`` on the website, etc.
    \item Create a public facing API that will potentially allow the service to be expanded with a mobile application.
\end{itemize}

\section{Stakeholders}
\begin{description}
    \item[Team Leader] Ammar Askar
    \item[Product Owner] Ammar Askar
    \item[Scrum Master] Abhijeet Chakrabarti
    \item[Development Team] Brian Quinn, Eric Lee
    \item[Users] Anyone in the world with access to the Internet
\end{description}


\section{Deliverables}
\subsection{Major Highlights:}
\begin{enumerate}
    \item {A web app which allows users to listen to music with their friends in real-time.}
    \item {A RESTful API allowing uploading of music, real time co-ordination and providing the back-end data for the website}
    \item {An integrated chat-system where users can chat with each other}
    \item {A social media platform to share music, post updates, and so on.}
\end{enumerate}

\subsection{Platform / Frameworks}
\begin{description}
    \item [Front-End] Bootstrap, jQuery, socket.io
    \item [Back-End] node.js, socket.io
\end{description}

\end{document}